\documentclass[]{article}
\usepackage{lmodern}
\usepackage{amssymb,amsmath}
\usepackage{ifxetex,ifluatex}
\usepackage{fixltx2e} % provides \textsubscript
\ifnum 0\ifxetex 1\fi\ifluatex 1\fi=0 % if pdftex
  \usepackage[T1]{fontenc}
  \usepackage[utf8]{inputenc}
\else % if luatex or xelatex
  \ifxetex
    \usepackage{mathspec}
  \else
    \usepackage{fontspec}
  \fi
  \defaultfontfeatures{Ligatures=TeX,Scale=MatchLowercase}
\fi
% use upquote if available, for straight quotes in verbatim environments
\IfFileExists{upquote.sty}{\usepackage{upquote}}{}
% use microtype if available
\IfFileExists{microtype.sty}{%
\usepackage{microtype}
\UseMicrotypeSet[protrusion]{basicmath} % disable protrusion for tt fonts
}{}
\usepackage[margin=1in]{geometry}
\usepackage{hyperref}
\hypersetup{unicode=true,
            pdfborder={0 0 0},
            breaklinks=true}
\urlstyle{same}  % don't use monospace font for urls
\usepackage{color}
\usepackage{fancyvrb}
\newcommand{\VerbBar}{|}
\newcommand{\VERB}{\Verb[commandchars=\\\{\}]}
\DefineVerbatimEnvironment{Highlighting}{Verbatim}{commandchars=\\\{\}}
% Add ',fontsize=\small' for more characters per line
\usepackage{framed}
\definecolor{shadecolor}{RGB}{248,248,248}
\newenvironment{Shaded}{\begin{snugshade}}{\end{snugshade}}
\newcommand{\AlertTok}[1]{\textcolor[rgb]{0.94,0.16,0.16}{#1}}
\newcommand{\AnnotationTok}[1]{\textcolor[rgb]{0.56,0.35,0.01}{\textbf{\textit{#1}}}}
\newcommand{\AttributeTok}[1]{\textcolor[rgb]{0.77,0.63,0.00}{#1}}
\newcommand{\BaseNTok}[1]{\textcolor[rgb]{0.00,0.00,0.81}{#1}}
\newcommand{\BuiltInTok}[1]{#1}
\newcommand{\CharTok}[1]{\textcolor[rgb]{0.31,0.60,0.02}{#1}}
\newcommand{\CommentTok}[1]{\textcolor[rgb]{0.56,0.35,0.01}{\textit{#1}}}
\newcommand{\CommentVarTok}[1]{\textcolor[rgb]{0.56,0.35,0.01}{\textbf{\textit{#1}}}}
\newcommand{\ConstantTok}[1]{\textcolor[rgb]{0.00,0.00,0.00}{#1}}
\newcommand{\ControlFlowTok}[1]{\textcolor[rgb]{0.13,0.29,0.53}{\textbf{#1}}}
\newcommand{\DataTypeTok}[1]{\textcolor[rgb]{0.13,0.29,0.53}{#1}}
\newcommand{\DecValTok}[1]{\textcolor[rgb]{0.00,0.00,0.81}{#1}}
\newcommand{\DocumentationTok}[1]{\textcolor[rgb]{0.56,0.35,0.01}{\textbf{\textit{#1}}}}
\newcommand{\ErrorTok}[1]{\textcolor[rgb]{0.64,0.00,0.00}{\textbf{#1}}}
\newcommand{\ExtensionTok}[1]{#1}
\newcommand{\FloatTok}[1]{\textcolor[rgb]{0.00,0.00,0.81}{#1}}
\newcommand{\FunctionTok}[1]{\textcolor[rgb]{0.00,0.00,0.00}{#1}}
\newcommand{\ImportTok}[1]{#1}
\newcommand{\InformationTok}[1]{\textcolor[rgb]{0.56,0.35,0.01}{\textbf{\textit{#1}}}}
\newcommand{\KeywordTok}[1]{\textcolor[rgb]{0.13,0.29,0.53}{\textbf{#1}}}
\newcommand{\NormalTok}[1]{#1}
\newcommand{\OperatorTok}[1]{\textcolor[rgb]{0.81,0.36,0.00}{\textbf{#1}}}
\newcommand{\OtherTok}[1]{\textcolor[rgb]{0.56,0.35,0.01}{#1}}
\newcommand{\PreprocessorTok}[1]{\textcolor[rgb]{0.56,0.35,0.01}{\textit{#1}}}
\newcommand{\RegionMarkerTok}[1]{#1}
\newcommand{\SpecialCharTok}[1]{\textcolor[rgb]{0.00,0.00,0.00}{#1}}
\newcommand{\SpecialStringTok}[1]{\textcolor[rgb]{0.31,0.60,0.02}{#1}}
\newcommand{\StringTok}[1]{\textcolor[rgb]{0.31,0.60,0.02}{#1}}
\newcommand{\VariableTok}[1]{\textcolor[rgb]{0.00,0.00,0.00}{#1}}
\newcommand{\VerbatimStringTok}[1]{\textcolor[rgb]{0.31,0.60,0.02}{#1}}
\newcommand{\WarningTok}[1]{\textcolor[rgb]{0.56,0.35,0.01}{\textbf{\textit{#1}}}}
\usepackage{longtable,booktabs}
\usepackage{graphicx,grffile}
\makeatletter
\def\maxwidth{\ifdim\Gin@nat@width>\linewidth\linewidth\else\Gin@nat@width\fi}
\def\maxheight{\ifdim\Gin@nat@height>\textheight\textheight\else\Gin@nat@height\fi}
\makeatother
% Scale images if necessary, so that they will not overflow the page
% margins by default, and it is still possible to overwrite the defaults
% using explicit options in \includegraphics[width, height, ...]{}
\setkeys{Gin}{width=\maxwidth,height=\maxheight,keepaspectratio}
\IfFileExists{parskip.sty}{%
\usepackage{parskip}
}{% else
\setlength{\parindent}{0pt}
\setlength{\parskip}{6pt plus 2pt minus 1pt}
}
\setlength{\emergencystretch}{3em}  % prevent overfull lines
\providecommand{\tightlist}{%
  \setlength{\itemsep}{0pt}\setlength{\parskip}{0pt}}
\setcounter{secnumdepth}{0}
% Redefines (sub)paragraphs to behave more like sections
\ifx\paragraph\undefined\else
\let\oldparagraph\paragraph
\renewcommand{\paragraph}[1]{\oldparagraph{#1}\mbox{}}
\fi
\ifx\subparagraph\undefined\else
\let\oldsubparagraph\subparagraph
\renewcommand{\subparagraph}[1]{\oldsubparagraph{#1}\mbox{}}
\fi

%%% Use protect on footnotes to avoid problems with footnotes in titles
\let\rmarkdownfootnote\footnote%
\def\footnote{\protect\rmarkdownfootnote}

%%% Change title format to be more compact
\usepackage{titling}

% Create subtitle command for use in maketitle
\providecommand{\subtitle}[1]{
  \posttitle{
    \begin{center}\large#1\end{center}
    }
}

\setlength{\droptitle}{-2em}

  \title{}
    \pretitle{\vspace{\droptitle}}
  \posttitle{}
    \author{}
    \preauthor{}\postauthor{}
    \date{}
    \predate{}\postdate{}
  

\begin{document}

class: center, middle

\hypertarget{programming-in-r}{%
\section{Programming in R}\label{programming-in-r}}

Adam Rawles

\begin{center}\rule{0.5\linewidth}{\linethickness}\end{center}

\hypertarget{recap}{%
\subsection{Recap}\label{recap}}

--

\begin{itemize}
\tightlist
\item
  Loading data from .csv and .xlsx
\end{itemize}

--

\begin{itemize}
\tightlist
\item
  Cleaning data

  \begin{itemize}
  \tightlist
  \item
    \texttt{is/as.xxxxxx()} functions
  \end{itemize}
\end{itemize}

--

\begin{itemize}
\tightlist
\item
  Summary statistics
\end{itemize}

--

\begin{itemize}
\tightlist
\item
  \texttt{mean()}, \texttt{median()}, \texttt{sd()}\ldots{}
\item
  \texttt{summary()}
\end{itemize}

--

\begin{itemize}
\tightlist
\item
  Plotting
\end{itemize}

--

\begin{itemize}
\tightlist
\item
  \texttt{plot()}
\item
  \texttt{hist()}
\end{itemize}

\begin{center}\rule{0.5\linewidth}{\linethickness}\end{center}

\hypertarget{overview}{%
\subsection{Overview}\label{overview}}

\begin{itemize}
\tightlist
\item
  User-defined functions
\end{itemize}

--

\begin{itemize}
\tightlist
\item
  For loops
\end{itemize}

--

\begin{itemize}
\tightlist
\item
  If/else statements
\end{itemize}

\begin{center}\rule{0.5\linewidth}{\linethickness}\end{center}

\hypertarget{functions}{%
\subsection{Functions}\label{functions}}

\begin{itemize}
\tightlist
\item
  Functions are how we perform any action in R
\end{itemize}

--

\begin{itemize}
\tightlist
\item
  We pass arguments to a function as an input, there is some form of
  transformation, and we get an output
\end{itemize}

--

\begin{itemize}
\tightlist
\item
  For example, \texttt{read.csv()} takes a .csv file, and turns it into
  a dataframe
\end{itemize}

--

\begin{itemize}
\tightlist
\item
  There are thousands of predefined functions in base R, and even more
  with packages
\end{itemize}

--

\begin{itemize}
\tightlist
\item
  But often, you'll need a specific function for a specific task\ldots{}
\end{itemize}

--

\begin{itemize}
\tightlist
\item
  And that's where user-defined functions come in
\end{itemize}

\begin{center}\rule{0.5\linewidth}{\linethickness}\end{center}

\hypertarget{user-defined-functions---basics}{%
\subsection{User-defined functions -
basics}\label{user-defined-functions---basics}}

\begin{itemize}
\tightlist
\item
  Functions can have multiple inputs, but only one output
\end{itemize}

--

\begin{itemize}
\tightlist
\item
  Functions are named*, and are always followed by () when used
\end{itemize}

--

\begin{itemize}
\tightlist
\item
  Functions names should be unique but memorable/logical
\end{itemize}

--

\begin{itemize}
\tightlist
\item
  Functions should be as simple and applicable as possible
\end{itemize}

--

\begin{itemize}
\tightlist
\item
  Functions will return the last evaluated (not assigned) variable, or
  whatever is included in the \texttt{return()} call
\end{itemize}

--

\begin{itemize}
\tightlist
\item
  Functions can only return \emph{one} thing (but vectors and lists
  still count as one thing)
\end{itemize}

--

* There are unnamed functions called ``anonymous functions'' but that's
for a different module

\begin{center}\rule{0.5\linewidth}{\linethickness}\end{center}

\hypertarget{user-defined-functions---structure}{%
\subsection{User-defined functions -
structure}\label{user-defined-functions---structure}}

\begin{itemize}
\tightlist
\item
  Functions are created with the following structure:
\end{itemize}

--

\begin{Shaded}
\begin{Highlighting}[]
\NormalTok{some_function <-}\StringTok{ }\ControlFlowTok{function}\NormalTok{(some_input, ...)\{}
\NormalTok{  operation}
  \KeywordTok{return}\NormalTok{(return_value)}
\NormalTok{\}}
\end{Highlighting}
\end{Shaded}

--

\begin{itemize}
\tightlist
\item
  We name the function, define what inputs we want, and then what we
  want to do with those inputs
\end{itemize}

\begin{center}\rule{0.5\linewidth}{\linethickness}\end{center}

\hypertarget{user-defined-functions---example}{%
\subsection{User-defined functions -
example}\label{user-defined-functions---example}}

--

\begin{Shaded}
\begin{Highlighting}[]
\NormalTok{sum_custom <-}\StringTok{ }\ControlFlowTok{function}\NormalTok{(x, y)\{}
\NormalTok{  new_value <-}\StringTok{ }\NormalTok{x }\OperatorTok{+}\StringTok{ }\NormalTok{y}
  \KeywordTok{return}\NormalTok{(new_value)}
\NormalTok{\}}

\KeywordTok{sum_custom}\NormalTok{(}\DecValTok{1}\NormalTok{,}\DecValTok{2}\NormalTok{)}
\end{Highlighting}
\end{Shaded}

\begin{verbatim}
## [1] 3
\end{verbatim}

\begin{center}\rule{0.5\linewidth}{\linethickness}\end{center}

\hypertarget{user-defined-functions---example-1}{%
\subsection{User-defined functions -
example}\label{user-defined-functions---example-1}}

This same function can also be written without an explicit
\texttt{return()} call:

--

\begin{Shaded}
\begin{Highlighting}[]
\NormalTok{sum_custom <-}\StringTok{ }\ControlFlowTok{function}\NormalTok{(x, y)\{}
\NormalTok{  x }\OperatorTok{+}\StringTok{ }\NormalTok{y}
\NormalTok{\}}

\KeywordTok{sum_custom}\NormalTok{(}\DecValTok{1}\NormalTok{,}\DecValTok{2}\NormalTok{)}
\end{Highlighting}
\end{Shaded}

\begin{verbatim}
## [1] 3
\end{verbatim}

\begin{center}\rule{0.5\linewidth}{\linethickness}\end{center}

\hypertarget{user-defined-functions---example-2}{%
\subsection{User-defined functions -
example}\label{user-defined-functions---example-2}}

--

\begin{itemize}
\tightlist
\item
  You can specify default values for any of your input parameters,
  making that argument optional:
\end{itemize}

--

\begin{Shaded}
\begin{Highlighting}[]
\NormalTok{combine_custom <-}\StringTok{ }\ControlFlowTok{function}\NormalTok{(string1, string2, }\DataTypeTok{delimiter =} \StringTok{" "}\NormalTok{) \{}
\NormalTok{  new_string <-}\StringTok{ }\KeywordTok{paste0}\NormalTok{(string1, delimiter, string2)}
  \KeywordTok{return}\NormalTok{(new_string)}
\NormalTok{\}}

\KeywordTok{combine_custom}\NormalTok{(}\DataTypeTok{string1 =} \StringTok{"hello"}\NormalTok{, }\DataTypeTok{string2 =} \StringTok{"world"}\NormalTok{)}
\end{Highlighting}
\end{Shaded}

\begin{verbatim}
## [1] "hello world"
\end{verbatim}

\begin{center}\rule{0.5\linewidth}{\linethickness}\end{center}

\hypertarget{user-defined-functions---example-3}{%
\subsection{User-defined functions -
example}\label{user-defined-functions---example-3}}

\begin{itemize}
\tightlist
\item
  But you can override the default by providing a value to that
  parameter when using the function
\end{itemize}

--

\begin{Shaded}
\begin{Highlighting}[]
\KeywordTok{combine_custom}\NormalTok{(}\DataTypeTok{string1 =} \StringTok{"hello"}\NormalTok{,}
               \DataTypeTok{string2 =} \StringTok{"world"}\NormalTok{,}
               \DataTypeTok{delimiter =} \StringTok{"!"}\NormalTok{)}
\end{Highlighting}
\end{Shaded}

\begin{verbatim}
## [1] "hello!world"
\end{verbatim}

\begin{center}\rule{0.5\linewidth}{\linethickness}\end{center}

\hypertarget{user-defined-functions---exercise}{%
\subsection{User-defined functions -
exercise}\label{user-defined-functions---exercise}}

\begin{itemize}
\tightlist
\item
  Option 1 (easy)

  \begin{itemize}
  \tightlist
  \item
    Write a function that takes 2 numbers, multiplies them together and
    divides the result by 2
  \end{itemize}
\end{itemize}

\begin{longtable}[]{@{}l@{}}
\toprule
\endhead
\begin{minipage}[t]{0.04\columnwidth}\raggedright
- Option 2 (intermediate) - Write a function that takes 2 vectors, and
multiplies the largest value of vector 1 by the smallest value in vector
2\strut
\end{minipage}\tabularnewline
\bottomrule
\end{longtable}

\begin{itemize}
\tightlist
\item
  Option 3 (advanced)

  \begin{itemize}
  \tightlist
  \item
    Write a function that takes 2 strings, and returns the 1st and 2nd
    values of each one (hint available)
  \end{itemize}
\end{itemize}

--

\begin{itemize}
\tightlist
\item
  Option 4 (theoretical)

  \begin{itemize}
  \tightlist
  \item
    Think of how you would write a function that can sum an indefinite
    number of numbers
  \end{itemize}
\end{itemize}

\begin{center}\rule{0.5\linewidth}{\linethickness}\end{center}

\hypertarget{user-defined-functions---answers}{%
\subsection{User-defined functions -
answers}\label{user-defined-functions---answers}}

\begin{itemize}
\tightlist
\item
  Option 1

  \begin{itemize}
  \tightlist
  \item
    Write a function that takes 2 numbers, multiplies them together and
    divides the result by 2
  \end{itemize}
\end{itemize}

\begin{Shaded}
\begin{Highlighting}[]
\NormalTok{option1_function <-}\StringTok{ }\ControlFlowTok{function}\NormalTok{(x,y)\{}
\NormalTok{  new_val <-}\StringTok{ }\NormalTok{(x }\OperatorTok{*}\StringTok{ }\NormalTok{y)}\OperatorTok{/}\DecValTok{2}
  \KeywordTok{return}\NormalTok{(new_val)}
\NormalTok{\}}

\KeywordTok{option1_function}\NormalTok{(}\DecValTok{4}\NormalTok{,}\DecValTok{4}\NormalTok{)}
\end{Highlighting}
\end{Shaded}

\begin{verbatim}
## [1] 8
\end{verbatim}

\begin{center}\rule{0.5\linewidth}{\linethickness}\end{center}

\hypertarget{user-defined-functions---answers-1}{%
\subsection{User-defined functions -
answers}\label{user-defined-functions---answers-1}}

\begin{itemize}
\tightlist
\item
  Option 2

  \begin{itemize}
  \tightlist
  \item
    Write a function that takes 2 vectors, and multiplies that largest
    value of vector 1 by the smallest value in vector 2
  \end{itemize}
\end{itemize}

\begin{Shaded}
\begin{Highlighting}[]
\NormalTok{option2_function <-}\StringTok{ }\ControlFlowTok{function}\NormalTok{(v1, v2)\{}
\NormalTok{  new_val <-}\StringTok{ }\KeywordTok{max}\NormalTok{(v1) }\OperatorTok{*}\StringTok{ }\KeywordTok{min}\NormalTok{(v2)}
  \KeywordTok{return}\NormalTok{(new_val)}
\NormalTok{\}}

\KeywordTok{option2_function}\NormalTok{(}\DataTypeTok{v1 =} \KeywordTok{c}\NormalTok{(}\DecValTok{1}\NormalTok{,}\DecValTok{2}\NormalTok{,}\DecValTok{3}\NormalTok{,}\DecValTok{4}\NormalTok{), }\DataTypeTok{v2 =} \KeywordTok{c}\NormalTok{(}\DecValTok{1}\NormalTok{,}\DecValTok{2}\NormalTok{,}\DecValTok{3}\NormalTok{,}\DecValTok{4}\NormalTok{,}\DecValTok{5}\NormalTok{))}
\end{Highlighting}
\end{Shaded}

\begin{verbatim}
## [1] 4
\end{verbatim}

\begin{center}\rule{0.5\linewidth}{\linethickness}\end{center}

\hypertarget{user-defined-functions---answers-2}{%
\subsection{User-defined functions -
answers}\label{user-defined-functions---answers-2}}

\begin{itemize}
\tightlist
\item
  Option 3

  \begin{itemize}
  \tightlist
  \item
    Write a function that takes 2 strings, and returns the 1st and 2nd
    values of each one
  \end{itemize}
\end{itemize}

\begin{Shaded}
\begin{Highlighting}[]
\NormalTok{option3_function <-}\StringTok{ }\ControlFlowTok{function}\NormalTok{(string1, string2)\{}
\NormalTok{  ret1 <-}\StringTok{ }\KeywordTok{substr}\NormalTok{(string1,}\DecValTok{1}\NormalTok{,}\DecValTok{2}\NormalTok{)}
\NormalTok{  ret2 <-}\StringTok{ }\KeywordTok{substr}\NormalTok{(string2, }\DecValTok{1}\NormalTok{,}\DecValTok{2}\NormalTok{)}
\NormalTok{  ret <-}\StringTok{ }\KeywordTok{c}\NormalTok{(ret1, ret2)}
  \KeywordTok{return}\NormalTok{(ret)}
\NormalTok{\}}

\KeywordTok{option3_function}\NormalTok{(}\StringTok{"hello"}\NormalTok{, }\StringTok{"world"}\NormalTok{)}
\end{Highlighting}
\end{Shaded}

\begin{verbatim}
## [1] "he" "wo"
\end{verbatim}

\begin{center}\rule{0.5\linewidth}{\linethickness}\end{center}

\hypertarget{user-defined-functions}{%
\subsection{User-defined functions}\label{user-defined-functions}}

\begin{itemize}
\tightlist
\item
  In some cases, you may want to provide a variable number of inputs to
  a function (like if you want to add an indefinite number of values
  together)
\end{itemize}

--

\begin{itemize}
\tightlist
\item
  For example, if you have a function that combines strings, you may
  want to accept any number of strings to combine
\end{itemize}

--

\begin{itemize}
\tightlist
\item
  To do this, we use the ellipsis (\ldots) argument when defining our
  function
\end{itemize}

\begin{Shaded}
\begin{Highlighting}[]
\NormalTok{some_function <-}\StringTok{ }\ControlFlowTok{function}\NormalTok{(...)\{}
\NormalTok{  arguments <-}\StringTok{ }\KeywordTok{list}\NormalTok{(...)}
  \KeywordTok{return}\NormalTok{(arguments)}
\NormalTok{\}}
\end{Highlighting}
\end{Shaded}

\begin{center}\rule{0.5\linewidth}{\linethickness}\end{center}

\hypertarget{user-defined-functions-1}{%
\subsection{User-defined functions}\label{user-defined-functions-1}}

\begin{Shaded}
\begin{Highlighting}[]
\KeywordTok{some_function}\NormalTok{(}\StringTok{"hello"}\NormalTok{, }\StringTok{"world"}\NormalTok{)}
\end{Highlighting}
\end{Shaded}

\begin{verbatim}
## [[1]]
## [1] "hello"
## 
## [[2]]
## [1] "world"
\end{verbatim}

\begin{Shaded}
\begin{Highlighting}[]
\KeywordTok{some_function}\NormalTok{(}\StringTok{"hello"}\NormalTok{, }\StringTok{"world"}\NormalTok{, }\StringTok{"again"}\NormalTok{)}
\end{Highlighting}
\end{Shaded}

\begin{verbatim}
## [[1]]
## [1] "hello"
## 
## [[2]]
## [1] "world"
## 
## [[3]]
## [1] "again"
\end{verbatim}

\begin{center}\rule{0.5\linewidth}{\linethickness}\end{center}

\hypertarget{user-defined-functions---environment}{%
\subsection{User-defined functions -
environment}\label{user-defined-functions---environment}}

--

\begin{itemize}
\tightlist
\item
  Functions have a local environment, meaning that anything calculated
  in the function is not accessible outside the function (except the
  value that is returned)
\end{itemize}

--

\begin{Shaded}
\begin{Highlighting}[]
\NormalTok{some_function <-}\StringTok{ }\ControlFlowTok{function}\NormalTok{(x, y)\{}
\NormalTok{  m <-}\StringTok{ }\NormalTok{x }\OperatorTok{*}\StringTok{ }\NormalTok{y}
\NormalTok{  s <-}\StringTok{ }\NormalTok{x }\OperatorTok{+}\StringTok{ }\NormalTok{y}
  \KeywordTok{return}\NormalTok{(m)}
\NormalTok{\}}

\KeywordTok{some_function}\NormalTok{(}\DecValTok{1}\NormalTok{, }\DecValTok{2}\NormalTok{)}
\end{Highlighting}
\end{Shaded}

\begin{verbatim}
## [1] 2
\end{verbatim}

\begin{Shaded}
\begin{Highlighting}[]
\KeywordTok{print}\NormalTok{(s)}
\end{Highlighting}
\end{Shaded}

\begin{verbatim}
## Error in print(s): object 's' not found
\end{verbatim}

--

\begin{itemize}
\item ~
  \hypertarget{this-code-errors-because-the-s-variable-isnt-accessible-outside-of-the-function}{%
  \subsection{\texorpdfstring{This code errors, because the \texttt{s}
  variable isn't accessible outside of the
  function}{This code errors, because the s variable isn't accessible outside of the function}}\label{this-code-errors-because-the-s-variable-isnt-accessible-outside-of-the-function}}
\end{itemize}

\hypertarget{user-defined-functions---returning}{%
\subsection{User-defined functions -
returning}\label{user-defined-functions---returning}}

\begin{itemize}
\tightlist
\item
  As previously mentioned, a function will return the last evaluated
  object, or whatever is returned via the \texttt{return()} function.
\end{itemize}

--

\begin{itemize}
\tightlist
\item
  What's really important to remember however, is that a function will
  return a \emph{copy} of the return value, not the return object
\end{itemize}

--

\begin{Shaded}
\begin{Highlighting}[]
\NormalTok{some_function <-}\StringTok{ }\ControlFlowTok{function}\NormalTok{(x) \{}
  \KeywordTok{return}\NormalTok{(x }\OperatorTok{+}\StringTok{ }\DecValTok{1}\NormalTok{)}
\NormalTok{\}}

\NormalTok{x <-}\StringTok{ }\DecValTok{1}
\KeywordTok{some_function}\NormalTok{(x)}
\end{Highlighting}
\end{Shaded}

\begin{verbatim}
## [1] 2
\end{verbatim}

\begin{Shaded}
\begin{Highlighting}[]
\NormalTok{x}
\end{Highlighting}
\end{Shaded}

\begin{verbatim}
## [1] 1
\end{verbatim}

\begin{center}\rule{0.5\linewidth}{\linethickness}\end{center}

\hypertarget{user-defined-functions---returning-1}{%
\subsection{User-defined functions -
returning}\label{user-defined-functions---returning-1}}

\begin{itemize}
\tightlist
\item
  To change the original object, we need to reassign the result of the
  function back to the object\ldots{}
\end{itemize}

--

\begin{Shaded}
\begin{Highlighting}[]
\NormalTok{x <-}\StringTok{ }\DecValTok{1}
\NormalTok{x <-}\StringTok{ }\KeywordTok{some_function}\NormalTok{(x)}
\NormalTok{x}
\end{Highlighting}
\end{Shaded}

\begin{verbatim}
## [1] 2
\end{verbatim}

\begin{center}\rule{0.5\linewidth}{\linethickness}\end{center}

\hypertarget{for-loops}{%
\subsection{For loops}\label{for-loops}}

--

\begin{itemize}
\tightlist
\item
  Sometimes, we may want to repeat the same action more than once
\end{itemize}

--

\begin{itemize}
\tightlist
\item
  For example, we might want a function to add 4 to every item in a
  vector, or get the mean for every column in a dataframe
\end{itemize}

--

\begin{itemize}
\tightlist
\item
  You could copy and paste the code required each time, or you could use
  a for loop
\end{itemize}

\begin{center}\rule{0.5\linewidth}{\linethickness}\end{center}

\hypertarget{for-loops-1}{%
\subsection{For loops}\label{for-loops-1}}

\begin{itemize}
\tightlist
\item
  With a for loop, you can iterate over ever item in a list or vector
  and perform an action
\end{itemize}

--

\begin{itemize}
\tightlist
\item
  While loops (which iterate until a condition is met) also exist, but
  we're going to focus on for loops
\end{itemize}

--

\begin{itemize}
\tightlist
\item
  For loops follow a basic structure:
\end{itemize}

\begin{Shaded}
\begin{Highlighting}[]
\ControlFlowTok{for}\NormalTok{ (identifier }\ControlFlowTok{in}\NormalTok{ list or vector)\{}
\NormalTok{  what we want to do with each item}
\NormalTok{\}}
\end{Highlighting}
\end{Shaded}

--

\begin{itemize}
\tightlist
\item
  The identifier becomes the variable name for accessing the current
  value in the body of the loop
\end{itemize}

--

\begin{itemize}
\tightlist
\item
  On each iteration, the identifier variable will take on a new value
\end{itemize}

\begin{center}\rule{0.5\linewidth}{\linethickness}\end{center}

\hypertarget{for-loops---structure}{%
\subsection{For loops - structure}\label{for-loops---structure}}

\begin{itemize}
\tightlist
\item
  You can also perform a for loop a defined number of times rather than
  iterating through a list/vector:
\end{itemize}

\begin{Shaded}
\begin{Highlighting}[]
\ControlFlowTok{for}\NormalTok{ (identifier }\ControlFlowTok{in} \KeywordTok{seq_along}\NormalTok{(}\DecValTok{1}\OperatorTok{:}\NormalTok{some_number))\{}
\NormalTok{  what we want to do with each item}
\NormalTok{\}}
\end{Highlighting}
\end{Shaded}

\begin{itemize}
\tightlist
\item
  In this case, the value of our identifier variable will change to the
  next number in our set of numbers
\end{itemize}

\begin{center}\rule{0.5\linewidth}{\linethickness}\end{center}

\hypertarget{for-loops---example}{%
\subsection{For loops - example}\label{for-loops---example}}

\begin{itemize}
\tightlist
\item
  Say we want to loop through a vector and print each value\ldots{}
\end{itemize}

--

\begin{Shaded}
\begin{Highlighting}[]
\NormalTok{vector1 <-}\StringTok{ }\KeywordTok{c}\NormalTok{(}\DecValTok{1}\NormalTok{,}\DecValTok{2}\NormalTok{,}\DecValTok{3}\NormalTok{,}\DecValTok{4}\NormalTok{,}\DecValTok{5}\NormalTok{,}\DecValTok{6}\NormalTok{,}\DecValTok{7}\NormalTok{,}\DecValTok{8}\NormalTok{)}
\ControlFlowTok{for}\NormalTok{ (i }\ControlFlowTok{in}\NormalTok{ vector1)\{}
  \KeywordTok{print}\NormalTok{(i)}
\NormalTok{\}}
\end{Highlighting}
\end{Shaded}

\begin{verbatim}
## [1] 1
## [1] 2
## [1] 3
## [1] 4
## [1] 5
## [1] 6
## [1] 7
## [1] 8
\end{verbatim}

\begin{center}\rule{0.5\linewidth}{\linethickness}\end{center}

\hypertarget{for-loops---excercise}{%
\subsection{For loops - excercise}\label{for-loops---excercise}}

\begin{itemize}
\tightlist
\item
  Option 1 (easy)

  \begin{itemize}
  \tightlist
  \item
    Write a for loop that divides each number in a vector of numbers by
    2
  \end{itemize}
\end{itemize}

--

\begin{itemize}
\tightlist
\item
  Option 2 (intermediate)

  \begin{itemize}
  \item ~
    \hypertarget{write-a-for-loop-that-produces-a-running-average-from-a-vector-of-numbers}{%
    \subsection{Write a for loop that produces a running average from a
    vector of
    numbers}\label{write-a-for-loop-that-produces-a-running-average-from-a-vector-of-numbers}}
  \end{itemize}
\item
  Option 3 (advanced)

  \begin{itemize}
  \tightlist
  \item
    Write a for loop that adds each value from one vector to the value
    at the next index in a second vector
  \end{itemize}
\end{itemize}

\begin{center}\rule{0.5\linewidth}{\linethickness}\end{center}

\hypertarget{for-loops---answers}{%
\subsection{For loops - answers}\label{for-loops---answers}}

\begin{itemize}
\tightlist
\item
  Option 1

  \begin{itemize}
  \tightlist
  \item
    Write a for loop that divides each number in a vector of numbers by
    2
  \end{itemize}
\end{itemize}

\begin{Shaded}
\begin{Highlighting}[]
\NormalTok{vector1 <-}\StringTok{ }\KeywordTok{c}\NormalTok{(}\DecValTok{2}\NormalTok{,}\DecValTok{4}\NormalTok{,}\DecValTok{6}\NormalTok{,}\DecValTok{8}\NormalTok{,}\DecValTok{10}\NormalTok{)}
\ControlFlowTok{for}\NormalTok{ (i }\ControlFlowTok{in}\NormalTok{ vector1)\{}
  \KeywordTok{print}\NormalTok{(i}\OperatorTok{/}\DecValTok{2}\NormalTok{)}
\NormalTok{\}}
\end{Highlighting}
\end{Shaded}

\begin{verbatim}
## [1] 1
## [1] 2
## [1] 3
## [1] 4
## [1] 5
\end{verbatim}

\begin{center}\rule{0.5\linewidth}{\linethickness}\end{center}

\hypertarget{for-loops---answers-1}{%
\subsection{For loops - answers}\label{for-loops---answers-1}}

\begin{itemize}
\tightlist
\item
  Option 2

  \begin{itemize}
  \tightlist
  \item
    Write a for loop that produces a running average from a vector of
    numbers
  \end{itemize}
\end{itemize}

\begin{Shaded}
\begin{Highlighting}[]
\NormalTok{vector1 <-}\StringTok{ }\KeywordTok{c}\NormalTok{(}\DecValTok{10}\NormalTok{,}\DecValTok{20}\NormalTok{,}\DecValTok{30}\NormalTok{,}\DecValTok{40}\NormalTok{,}\DecValTok{70}\NormalTok{,}\DecValTok{100}\NormalTok{)}
\NormalTok{total <-}\StringTok{ }\DecValTok{0}
\NormalTok{counter <-}\StringTok{ }\DecValTok{0}
\ControlFlowTok{for}\NormalTok{ (i }\ControlFlowTok{in}\NormalTok{ vector1)\{}
\NormalTok{  counter <-}\StringTok{ }\NormalTok{counter }\OperatorTok{+}\StringTok{ }\DecValTok{1}
\NormalTok{  total <-}\StringTok{ }\NormalTok{total }\OperatorTok{+}\StringTok{ }\NormalTok{i}
 \KeywordTok{print}\NormalTok{(total}\OperatorTok{/}\NormalTok{counter)}
\NormalTok{\}}
\end{Highlighting}
\end{Shaded}

\begin{verbatim}
## [1] 10
## [1] 15
## [1] 20
## [1] 25
## [1] 34
## [1] 45
\end{verbatim}

\begin{center}\rule{0.5\linewidth}{\linethickness}\end{center}

\hypertarget{for-loops---answers-2}{%
\subsection{For loops - answers}\label{for-loops---answers-2}}

\begin{itemize}
\tightlist
\item
  Option 3

  \begin{itemize}
  \tightlist
  \item
    Write a for loop that adds each value from one vector to the value
    at the next index in a second vector
  \end{itemize}
\end{itemize}

\begin{Shaded}
\begin{Highlighting}[]
\NormalTok{vector1 <-}\StringTok{ }\KeywordTok{c}\NormalTok{(}\DecValTok{1}\NormalTok{,}\DecValTok{5}\NormalTok{,}\DecValTok{10}\NormalTok{,}\DecValTok{15}\NormalTok{)}
\NormalTok{vector2 <-}\StringTok{ }\KeywordTok{c}\NormalTok{(}\DecValTok{2}\NormalTok{,}\DecValTok{6}\NormalTok{,}\DecValTok{10}\NormalTok{,}\DecValTok{14}\NormalTok{,}\DecValTok{18}\NormalTok{)}

\ControlFlowTok{for}\NormalTok{ (i }\ControlFlowTok{in} \KeywordTok{seq_along}\NormalTok{(vector))\{}
  \KeywordTok{print}\NormalTok{(vector1[i] }\OperatorTok{+}\StringTok{ }\NormalTok{vector2[i}\OperatorTok{+}\DecValTok{1}\NormalTok{])}
\NormalTok{\}}
\end{Highlighting}
\end{Shaded}

\begin{verbatim}
## [1] 7
\end{verbatim}

\begin{center}\rule{0.5\linewidth}{\linethickness}\end{center}

\hypertarget{if-else-statements}{%
\subsection{If else statements}\label{if-else-statements}}

\begin{itemize}
\tightlist
\item
  Sometimes, you'll only want to perform an action if a certain criteria
  is met
\end{itemize}

--

\begin{itemize}
\tightlist
\item
  For example, you may only want to add 4 to a number if it's greater
  than 10
\end{itemize}

--

\begin{itemize}
\tightlist
\item
  To perform a certain action based on mutliple criteria, you use an if
  else statement
\end{itemize}

\begin{center}\rule{0.5\linewidth}{\linethickness}\end{center}

\hypertarget{if-else-statements---structure}{%
\subsection{If else statements -
structure}\label{if-else-statements---structure}}

\begin{itemize}
\tightlist
\item
  There are 3 main `types' of if else statements
\end{itemize}

--

\begin{itemize}
\tightlist
\item
  Simple if statements

  \begin{itemize}
  \tightlist
  \item
    If the criteria is fulfilled, perform the action, otherwise do
    nothing:
  \end{itemize}
\end{itemize}

--

\begin{Shaded}
\begin{Highlighting}[]
\ControlFlowTok{if}\NormalTok{ (criteria)\{}
\NormalTok{  do something}
\NormalTok{\}}
\end{Highlighting}
\end{Shaded}

--

\begin{itemize}
\tightlist
\item
  If else statements

  \begin{itemize}
  \tightlist
  \item
    If the criteria is fulfilled, perform the action, otherwise do
    something else:
  \end{itemize}
\end{itemize}

--

\begin{Shaded}
\begin{Highlighting}[]
\ControlFlowTok{if}\NormalTok{ (criteria)\{}
\NormalTok{  do something}
\NormalTok{\} }\ControlFlowTok{else}\NormalTok{ \{}
\NormalTok{  do something }\ControlFlowTok{else}
\NormalTok{\}}
\end{Highlighting}
\end{Shaded}

\begin{center}\rule{0.5\linewidth}{\linethickness}\end{center}

\hypertarget{if-else-statements---structure-1}{%
\subsection{If else statements -
structure}\label{if-else-statements---structure-1}}

\begin{itemize}
\tightlist
\item
  If and if else statements

  \begin{itemize}
  \tightlist
  \item
    If the criteria is fulfilled perform the action, otherwise if a
    different criteria is fulfilled do something else, otherwise do
    nothing:
  \end{itemize}
\end{itemize}

--

\begin{Shaded}
\begin{Highlighting}[]
\ControlFlowTok{if}\NormalTok{ (criteria)\{}
\NormalTok{  do something}
\NormalTok{\} }\ControlFlowTok{else} \ControlFlowTok{if}\NormalTok{ (other criteria) \{}
\NormalTok{  do something }\ControlFlowTok{else}
\NormalTok{\}}
\end{Highlighting}
\end{Shaded}

\begin{center}\rule{0.5\linewidth}{\linethickness}\end{center}

\hypertarget{if-else-statements---criteria}{%
\subsection{If else statements -
criteria}\label{if-else-statements---criteria}}

\begin{itemize}
\tightlist
\item
  You can include multiple criteria in one if or else statement\ldots{}
\end{itemize}

--

\begin{Shaded}
\begin{Highlighting}[]
\ControlFlowTok{if}\NormalTok{ (criteria }\DecValTok{1} \OperatorTok{|}\StringTok{ }\NormalTok{criteria }\DecValTok{2}\NormalTok{)\{}
  
\NormalTok{\}}

\ControlFlowTok{if}\NormalTok{ (criteria }\DecValTok{1} \OperatorTok{&}\StringTok{ }\NormalTok{criteria }\DecValTok{2}\NormalTok{)\{}
  
\NormalTok{\}}
\end{Highlighting}
\end{Shaded}

\begin{center}\rule{0.5\linewidth}{\linethickness}\end{center}

\hypertarget{if-else-statements---example}{%
\subsection{If else statements -
example}\label{if-else-statements---example}}

--

\begin{Shaded}
\begin{Highlighting}[]
\NormalTok{vector1 <-}\StringTok{ }\KeywordTok{c}\NormalTok{(}\DecValTok{1}\NormalTok{,}\DecValTok{2}\NormalTok{,}\DecValTok{3}\NormalTok{,}\DecValTok{4}\NormalTok{,}\DecValTok{5}\NormalTok{)}

\ControlFlowTok{for}\NormalTok{ (i }\ControlFlowTok{in}\NormalTok{ vector1)\{}
  \ControlFlowTok{if}\NormalTok{ (i }\OperatorTok{==}\StringTok{ }\DecValTok{1}\NormalTok{) \{}
    \KeywordTok{print}\NormalTok{(}\StringTok{"The value is 1"}\NormalTok{)}
\NormalTok{  \} }\ControlFlowTok{else} \ControlFlowTok{if}\NormalTok{ (i }\OperatorTok{==}\StringTok{ }\DecValTok{2}\NormalTok{) \{}
    \KeywordTok{print}\NormalTok{(}\StringTok{"The value is 2"}\NormalTok{)}
\NormalTok{  \} }\ControlFlowTok{else} \ControlFlowTok{if}\NormalTok{ (i }\OperatorTok{==}\StringTok{ }\DecValTok{3} \OperatorTok{|}\StringTok{ }\NormalTok{i }\OperatorTok{==}\StringTok{ }\DecValTok{4}\NormalTok{)\{}
    \KeywordTok{print}\NormalTok{(}\StringTok{"The value is 3 or 4"}\NormalTok{)}
\NormalTok{  \} }\ControlFlowTok{else}\NormalTok{ \{}
    \KeywordTok{print}\NormalTok{(}\StringTok{"The value is not 1, 2, 3 or 4"}\NormalTok{)}
\NormalTok{  \}}
\NormalTok{\}}
\end{Highlighting}
\end{Shaded}

\begin{verbatim}
## [1] "The value is 1"
## [1] "The value is 2"
## [1] "The value is 3 or 4"
## [1] "The value is 3 or 4"
## [1] "The value is not 1, 2, 3 or 4"
\end{verbatim}

\begin{center}\rule{0.5\linewidth}{\linethickness}\end{center}

\hypertarget{final---exercise}{%
\subsection{Final - exercise}\label{final---exercise}}

\begin{itemize}
\tightlist
\item
  Option 1 (easy)

  \begin{itemize}
  \tightlist
  \item
    Write a function that loops through a vector of numbers and returns
    only the even ones
  \item
    Hint: you can use x \%\% y to check if a number is even
  \end{itemize}
\end{itemize}

--

\begin{itemize}
\tightlist
\item
  Option 2 (intermediate)

  \begin{itemize}
  \tightlist
  \item
    Write a function that loops through a vector of numbers and returns
    the values that are smaller than the value at the same index in a
    second vector
  \item
    You can assume that the two vectors will always be the same length
  \end{itemize}
\end{itemize}

--

\begin{itemize}
\tightlist
\item
  Option 3 (advanced)

  \begin{itemize}
  \tightlist
  \item
    Write a function that loops through a vector of numbers and square
    it if it is a multiple of 4, otherwise replace the value with the
    previous value in the vector or 0 if the value is first in the
    vector and return
  \item
    Hint: use x / y \%\% 2 to check for multiples
  \end{itemize}
\end{itemize}

\begin{center}\rule{0.5\linewidth}{\linethickness}\end{center}

\hypertarget{final---answers}{%
\subsection{Final - answers}\label{final---answers}}

\begin{itemize}
\tightlist
\item
  Option 1

  \begin{itemize}
  \tightlist
  \item
    Write a function that loops through a vector of numbers and returns
    only the even ones
  \end{itemize}
\end{itemize}

\begin{Shaded}
\begin{Highlighting}[]
\NormalTok{option1_function <-}\StringTok{ }\ControlFlowTok{function}\NormalTok{(v) \{}
\NormalTok{  return_vector <-}\StringTok{ }\KeywordTok{c}\NormalTok{()}
  \ControlFlowTok{for}\NormalTok{ (i }\ControlFlowTok{in}\NormalTok{ v)\{}
    \ControlFlowTok{if}\NormalTok{ (i }\OperatorTok\StringTok{ }\DecValTok{2} \OperatorTok{==}\StringTok{ }\DecValTok{0}\NormalTok{)\{}
\NormalTok{      return_vector <-}\StringTok{ }\KeywordTok{append}\NormalTok{(return_vector, i)}
\NormalTok{    \}}
\NormalTok{  \}}
\NormalTok{\}}

\KeywordTok{option1_function}\NormalTok{(}\DataTypeTok{v =} \KeywordTok{c}\NormalTok{(}\DecValTok{1}\NormalTok{,}\DecValTok{2}\NormalTok{,}\DecValTok{3}\NormalTok{,}\DecValTok{4}\NormalTok{,}\DecValTok{5}\NormalTok{,}\DecValTok{6}\NormalTok{,}\DecValTok{7}\NormalTok{,}\DecValTok{8}\NormalTok{))}
\end{Highlighting}
\end{Shaded}

\begin{center}\rule{0.5\linewidth}{\linethickness}\end{center}

\hypertarget{final---answers-1}{%
\subsection{Final - answers}\label{final---answers-1}}

\begin{itemize}
\tightlist
\item
  Option 2

  \begin{itemize}
  \tightlist
  \item
    Write a function that loops through a vector of numbers and returns
    the values that are smaller than the value at the same index in a
    second vector
  \end{itemize}
\end{itemize}

\begin{Shaded}
\begin{Highlighting}[]
\NormalTok{option2_function <-}\StringTok{ }\ControlFlowTok{function}\NormalTok{(v1, v2) \{}
\NormalTok{  return_vector <-}\StringTok{ }\KeywordTok{c}\NormalTok{()}
  \ControlFlowTok{for}\NormalTok{ (i }\ControlFlowTok{in} \KeywordTok{seq_along}\NormalTok{(v1))\{}
    \ControlFlowTok{if}\NormalTok{ (v1[i] }\OperatorTok{<}\StringTok{ }\NormalTok{v2[i])\{}
\NormalTok{      return_vector <-}\StringTok{ }\KeywordTok{append}\NormalTok{(return_vector, v1[i])}
\NormalTok{      \}}
\NormalTok{  \}}
\NormalTok{\}}

\KeywordTok{option2_function}\NormalTok{(}\DataTypeTok{v1 =} \KeywordTok{c}\NormalTok{(}\DecValTok{1}\NormalTok{,}\DecValTok{2}\NormalTok{,}\DecValTok{3}\NormalTok{,}\DecValTok{4}\NormalTok{,}\DecValTok{5}\NormalTok{,}\DecValTok{6}\NormalTok{,}\DecValTok{7}\NormalTok{,}\DecValTok{8}\NormalTok{), }\DataTypeTok{v2 =} \KeywordTok{c}\NormalTok{(}\DecValTok{2}\NormalTok{,}\DecValTok{1}\NormalTok{,}\DecValTok{4}\NormalTok{,}\DecValTok{5}\NormalTok{,}\DecValTok{2}\NormalTok{,}\DecValTok{1}\NormalTok{,}\DecValTok{1}\NormalTok{,}\DecValTok{1}\NormalTok{))}
\end{Highlighting}
\end{Shaded}

\begin{center}\rule{0.5\linewidth}{\linethickness}\end{center}

\hypertarget{final---answers-2}{%
\subsection{Final - answers}\label{final---answers-2}}

\begin{itemize}
\tightlist
\item
  Option 3

  \begin{itemize}
  \tightlist
  \item
    Write a function that loops through a vector of numbers and squares
    it if it is a multiple of 4, otherwise replace the value with the
    previous value in the vector or 0 if the value is first in the
    vector and return
  \end{itemize}
\end{itemize}

\begin{center}\rule{0.5\linewidth}{\linethickness}\end{center}

\hypertarget{final---answers-3}{%
\subsection{Final - answers}\label{final---answers-3}}

\begin{Shaded}
\begin{Highlighting}[]
\NormalTok{option3_function <-}\StringTok{ }\ControlFlowTok{function}\NormalTok{(v) \{}
  \ControlFlowTok{for}\NormalTok{ (i }\ControlFlowTok{in} \KeywordTok{seq_along}\NormalTok{(v))\{}
    \ControlFlowTok{if}\NormalTok{ ((v[i] }\OperatorTok{/}\StringTok{ }\DecValTok{4}\NormalTok{) }\OperatorTok\StringTok{ }\DecValTok{2} \OperatorTok{==}\StringTok{ }\DecValTok{0}\NormalTok{)\{}
\NormalTok{      v[i] <-}\StringTok{ }\NormalTok{v[i]}\OperatorTok{^}\DecValTok{2}
\NormalTok{    \} }\ControlFlowTok{else}\NormalTok{ \{}
        \ControlFlowTok{if}\NormalTok{ (i }\OperatorTok{==}\StringTok{ }\DecValTok{1}\NormalTok{)\{}
\NormalTok{         v[i] =}\StringTok{ }\DecValTok{0} 
\NormalTok{        \} }\ControlFlowTok{else}\NormalTok{ \{}
\NormalTok{          v[i] =}\StringTok{ }\NormalTok{v[i}\DecValTok{-1}\NormalTok{]}
\NormalTok{        \}}
\NormalTok{    \}}
\NormalTok{  \}}
  \KeywordTok{return}\NormalTok{(v)}
\NormalTok{\}}

\KeywordTok{option3_function}\NormalTok{(}\DataTypeTok{v =} \KeywordTok{c}\NormalTok{(}\DecValTok{1}\NormalTok{,}\DecValTok{2}\NormalTok{,}\DecValTok{3}\NormalTok{,}\DecValTok{4}\NormalTok{,}\DecValTok{5}\NormalTok{,}\DecValTok{6}\NormalTok{,}\DecValTok{7}\NormalTok{,}\DecValTok{8}\NormalTok{))}
\end{Highlighting}
\end{Shaded}

\begin{verbatim}
## [1]  0  0  0  0  0  0  0 64
\end{verbatim}

\begin{center}\rule{0.5\linewidth}{\linethickness}\end{center}

\hypertarget{conclusion}{%
\subsection{Conclusion}\label{conclusion}}

--

\begin{itemize}
\tightlist
\item
  User-defined functions

  \begin{itemize}
  \tightlist
  \item
    We use functions to perform repeatable and generalizable tasks
  \end{itemize}
\end{itemize}

--

\begin{itemize}
\tightlist
\item
  For loops

  \begin{itemize}
  \tightlist
  \item
    We use for loops to iterate over vectors/lists, or to perform an
    action a certain number of times
  \end{itemize}
\end{itemize}

--

\begin{itemize}
\tightlist
\item
  If else statements

  \begin{itemize}
  \tightlist
  \item
    With if, else if, and else statements, we can perform actions only
    when a certain criteria is met
  \end{itemize}
\end{itemize}

\begin{center}\rule{0.5\linewidth}{\linethickness}\end{center}

\hypertarget{future-modules-optional}{%
\subsection{Future modules (optional)}\label{future-modules-optional}}

\begin{itemize}
\tightlist
\item
  Statistical analysis
\item
  Simulations
\item
  Improving efficiency
\end{itemize}


\end{document}
