\documentclass[]{article}
\usepackage{lmodern}
\usepackage{amssymb,amsmath}
\usepackage{ifxetex,ifluatex}
\usepackage{fixltx2e} % provides \textsubscript
\ifnum 0\ifxetex 1\fi\ifluatex 1\fi=0 % if pdftex
  \usepackage[T1]{fontenc}
  \usepackage[utf8]{inputenc}
\else % if luatex or xelatex
  \ifxetex
    \usepackage{mathspec}
  \else
    \usepackage{fontspec}
  \fi
  \defaultfontfeatures{Ligatures=TeX,Scale=MatchLowercase}
\fi
% use upquote if available, for straight quotes in verbatim environments
\IfFileExists{upquote.sty}{\usepackage{upquote}}{}
% use microtype if available
\IfFileExists{microtype.sty}{%
\usepackage{microtype}
\UseMicrotypeSet[protrusion]{basicmath} % disable protrusion for tt fonts
}{}
\usepackage[margin=1in]{geometry}
\usepackage{hyperref}
\hypersetup{unicode=true,
            pdftitle={Intro to R},
            pdfborder={0 0 0},
            breaklinks=true}
\urlstyle{same}  % don't use monospace font for urls
\usepackage{color}
\usepackage{fancyvrb}
\newcommand{\VerbBar}{|}
\newcommand{\VERB}{\Verb[commandchars=\\\{\}]}
\DefineVerbatimEnvironment{Highlighting}{Verbatim}{commandchars=\\\{\}}
% Add ',fontsize=\small' for more characters per line
\usepackage{framed}
\definecolor{shadecolor}{RGB}{248,248,248}
\newenvironment{Shaded}{\begin{snugshade}}{\end{snugshade}}
\newcommand{\AlertTok}[1]{\textcolor[rgb]{0.94,0.16,0.16}{#1}}
\newcommand{\AnnotationTok}[1]{\textcolor[rgb]{0.56,0.35,0.01}{\textbf{\textit{#1}}}}
\newcommand{\AttributeTok}[1]{\textcolor[rgb]{0.77,0.63,0.00}{#1}}
\newcommand{\BaseNTok}[1]{\textcolor[rgb]{0.00,0.00,0.81}{#1}}
\newcommand{\BuiltInTok}[1]{#1}
\newcommand{\CharTok}[1]{\textcolor[rgb]{0.31,0.60,0.02}{#1}}
\newcommand{\CommentTok}[1]{\textcolor[rgb]{0.56,0.35,0.01}{\textit{#1}}}
\newcommand{\CommentVarTok}[1]{\textcolor[rgb]{0.56,0.35,0.01}{\textbf{\textit{#1}}}}
\newcommand{\ConstantTok}[1]{\textcolor[rgb]{0.00,0.00,0.00}{#1}}
\newcommand{\ControlFlowTok}[1]{\textcolor[rgb]{0.13,0.29,0.53}{\textbf{#1}}}
\newcommand{\DataTypeTok}[1]{\textcolor[rgb]{0.13,0.29,0.53}{#1}}
\newcommand{\DecValTok}[1]{\textcolor[rgb]{0.00,0.00,0.81}{#1}}
\newcommand{\DocumentationTok}[1]{\textcolor[rgb]{0.56,0.35,0.01}{\textbf{\textit{#1}}}}
\newcommand{\ErrorTok}[1]{\textcolor[rgb]{0.64,0.00,0.00}{\textbf{#1}}}
\newcommand{\ExtensionTok}[1]{#1}
\newcommand{\FloatTok}[1]{\textcolor[rgb]{0.00,0.00,0.81}{#1}}
\newcommand{\FunctionTok}[1]{\textcolor[rgb]{0.00,0.00,0.00}{#1}}
\newcommand{\ImportTok}[1]{#1}
\newcommand{\InformationTok}[1]{\textcolor[rgb]{0.56,0.35,0.01}{\textbf{\textit{#1}}}}
\newcommand{\KeywordTok}[1]{\textcolor[rgb]{0.13,0.29,0.53}{\textbf{#1}}}
\newcommand{\NormalTok}[1]{#1}
\newcommand{\OperatorTok}[1]{\textcolor[rgb]{0.81,0.36,0.00}{\textbf{#1}}}
\newcommand{\OtherTok}[1]{\textcolor[rgb]{0.56,0.35,0.01}{#1}}
\newcommand{\PreprocessorTok}[1]{\textcolor[rgb]{0.56,0.35,0.01}{\textit{#1}}}
\newcommand{\RegionMarkerTok}[1]{#1}
\newcommand{\SpecialCharTok}[1]{\textcolor[rgb]{0.00,0.00,0.00}{#1}}
\newcommand{\SpecialStringTok}[1]{\textcolor[rgb]{0.31,0.60,0.02}{#1}}
\newcommand{\StringTok}[1]{\textcolor[rgb]{0.31,0.60,0.02}{#1}}
\newcommand{\VariableTok}[1]{\textcolor[rgb]{0.00,0.00,0.00}{#1}}
\newcommand{\VerbatimStringTok}[1]{\textcolor[rgb]{0.31,0.60,0.02}{#1}}
\newcommand{\WarningTok}[1]{\textcolor[rgb]{0.56,0.35,0.01}{\textbf{\textit{#1}}}}
\usepackage{graphicx,grffile}
\makeatletter
\def\maxwidth{\ifdim\Gin@nat@width>\linewidth\linewidth\else\Gin@nat@width\fi}
\def\maxheight{\ifdim\Gin@nat@height>\textheight\textheight\else\Gin@nat@height\fi}
\makeatother
% Scale images if necessary, so that they will not overflow the page
% margins by default, and it is still possible to overwrite the defaults
% using explicit options in \includegraphics[width, height, ...]{}
\setkeys{Gin}{width=\maxwidth,height=\maxheight,keepaspectratio}
\IfFileExists{parskip.sty}{%
\usepackage{parskip}
}{% else
\setlength{\parindent}{0pt}
\setlength{\parskip}{6pt plus 2pt minus 1pt}
}
\setlength{\emergencystretch}{3em}  % prevent overfull lines
\providecommand{\tightlist}{%
  \setlength{\itemsep}{0pt}\setlength{\parskip}{0pt}}
\setcounter{secnumdepth}{0}
% Redefines (sub)paragraphs to behave more like sections
\ifx\paragraph\undefined\else
\let\oldparagraph\paragraph
\renewcommand{\paragraph}[1]{\oldparagraph{#1}\mbox{}}
\fi
\ifx\subparagraph\undefined\else
\let\oldsubparagraph\subparagraph
\renewcommand{\subparagraph}[1]{\oldsubparagraph{#1}\mbox{}}
\fi

%%% Use protect on footnotes to avoid problems with footnotes in titles
\let\rmarkdownfootnote\footnote%
\def\footnote{\protect\rmarkdownfootnote}

%%% Change title format to be more compact
\usepackage{titling}

% Create subtitle command for use in maketitle
\providecommand{\subtitle}[1]{
  \posttitle{
    \begin{center}\large#1\end{center}
    }
}

\setlength{\droptitle}{-2em}

  \title{Intro to R}
    \pretitle{\vspace{\droptitle}\centering\huge}
  \posttitle{\par}
    \author{}
    \preauthor{}\postauthor{}
    \date{}
    \predate{}\postdate{}
  

\begin{document}
\maketitle

class: center, middle

\hypertarget{introduction-to-r}{%
\section{Introduction to R}\label{introduction-to-r}}

\hypertarget{adam-rawles}{%
\subsection{Adam Rawles}\label{adam-rawles}}

\hypertarget{overview}{%
\subsection{Overview}\label{overview}}

--

\begin{itemize}
\tightlist
\item
  What is R and RStudio?
\end{itemize}

--

\begin{itemize}
\tightlist
\item
  Basic arithmetic operators
\end{itemize}

--

\begin{itemize}
\tightlist
\item
  Variable assigment
\end{itemize}

--

\begin{itemize}
\tightlist
\item
  Data types
\end{itemize}

--

\begin{itemize}
\tightlist
\item
  Data structures (including subsetting)
\end{itemize}

--

\begin{itemize}
\tightlist
\item
  Functions
\end{itemize}

\begin{center}\rule{0.5\linewidth}{\linethickness}\end{center}

\hypertarget{learning-objectives}{%
\subsection{Learning Objectives}\label{learning-objectives}}

\begin{itemize}
\tightlist
\item
  Understand the basics of RStudio
\end{itemize}

--

\begin{itemize}
\tightlist
\item
  Know the difference between the data types and structures in R
\end{itemize}

--

\begin{itemize}
\tightlist
\item
  Understand the concepts of how functions operate in R
\end{itemize}

--

\begin{itemize}
\tightlist
\item
  Know where/how to look things up!
\end{itemize}

\begin{center}\rule{0.5\linewidth}{\linethickness}\end{center}

\hypertarget{what-is-r}{%
\subsection{What is R?}\label{what-is-r}}

\begin{itemize}
\tightlist
\item
  Background
\end{itemize}

--

\begin{itemize}
\tightlist
\item
  Public license programming language
\end{itemize}

--

\begin{itemize}
\tightlist
\item
  Used mostly for statistical computing
\end{itemize}

--

\begin{itemize}
\tightlist
\item
  Supported with many packages
\end{itemize}

--

\begin{itemize}
\tightlist
\item
  Interface
\end{itemize}

--

\begin{itemize}
\tightlist
\item
  R uses a command-line interface
\end{itemize}

--

\begin{itemize}
\tightlist
\item
  But there are many GUIs available (such as RStudio)
\end{itemize}

\begin{center}\rule{0.5\linewidth}{\linethickness}\end{center}

\hypertarget{rstudio}{%
\subsection{RStudio}\label{rstudio}}

\begin{itemize}
\tightlist
\item
  RStudio is one of the most popular R environments
\end{itemize}

--

\begin{itemize}
\tightlist
\item
  RStudio has 4 main panes:
\end{itemize}

\begin{center}\rule{0.5\linewidth}{\linethickness}\end{center}

\hypertarget{rstudio-1}{%
\subsection{RStudio}\label{rstudio-1}}

\begin{itemize}
\tightlist
\item
  Source
\end{itemize}

--

\begin{itemize}
\tightlist
\item
  This is where your scripts (pre-written lines of codes) are displayed
\end{itemize}

--

\begin{itemize}
\tightlist
\item
  If you click on a variable in the Environment pane, this will also be
  displayed in the source window
\end{itemize}

--

\begin{itemize}
\tightlist
\item
  Console
\end{itemize}

--

\begin{itemize}
\tightlist
\item
  This is where the commands are actually executed
\end{itemize}

--

\begin{itemize}
\tightlist
\item
  When you run a script from the source, it is passed to the console
  line by line
\end{itemize}

\begin{center}\rule{0.5\linewidth}{\linethickness}\end{center}

\hypertarget{rstudio-2}{%
\subsection{RStudio}\label{rstudio-2}}

\begin{itemize}
\tightlist
\item
  Environment
\end{itemize}

--

\begin{itemize}
\tightlist
\item
  This pane displays any variables or data structures you've created
\end{itemize}

--

\begin{itemize}
\tightlist
\item
  This pane will also display any custom functions you've created, which
  we'll move onto in later modules
\end{itemize}

--

\begin{itemize}
\tightlist
\item
  History/Files/Plots/Help
\end{itemize}

--

\begin{itemize}
\tightlist
\item
  This pane can display a number of things
\end{itemize}

--

\begin{itemize}
\tightlist
\item
  The most useful of which are the Files tab, which displays all the
  files in the folder you're working in and the Plots tab, which will
  show any plots that you've created
\end{itemize}

--

Note: You can move which pane shows which windows via the Tools
-\textgreater{} Global Options dialog

\begin{center}\rule{0.5\linewidth}{\linethickness}\end{center}

\hypertarget{basic-arithmetic}{%
\subsection{Basic arithmetic}\label{basic-arithmetic}}

\begin{itemize}
\tightlist
\item
  The basic arithmetic operators in R are very similar to Excel
\end{itemize}

--

\begin{Shaded}
\begin{Highlighting}[]
\DecValTok{2} \OperatorTok{+}\StringTok{ }\DecValTok{2} \CommentTok{#add}
\end{Highlighting}
\end{Shaded}

\begin{verbatim}
## [1] 4
\end{verbatim}

\begin{Shaded}
\begin{Highlighting}[]
\DecValTok{2} \OperatorTok{*}\StringTok{ }\DecValTok{3} \CommentTok{#multiply}
\end{Highlighting}
\end{Shaded}

\begin{verbatim}
## [1] 6
\end{verbatim}

--

\begin{Shaded}
\begin{Highlighting}[]
\DecValTok{6}\OperatorTok{/}\DecValTok{2} \CommentTok{#divide}
\end{Highlighting}
\end{Shaded}

\begin{verbatim}
## [1] 3
\end{verbatim}

\begin{Shaded}
\begin{Highlighting}[]
\DecValTok{10} \OperatorTok{-}\StringTok{ }\DecValTok{5} \CommentTok{#subtract}
\end{Highlighting}
\end{Shaded}

\begin{verbatim}
## [1] 5
\end{verbatim}

\begin{center}\rule{0.5\linewidth}{\linethickness}\end{center}

\hypertarget{logical-operators}{%
\subsection{Logical operators}\label{logical-operators}}

\begin{itemize}
\tightlist
\item
  Logical operators in R are a bit different
\end{itemize}

--

\begin{Shaded}
\begin{Highlighting}[]
\DecValTok{1} \OperatorTok{==}\StringTok{ }\DecValTok{1} \CommentTok{#equal}
\end{Highlighting}
\end{Shaded}

\begin{verbatim}
## [1] TRUE
\end{verbatim}

\begin{Shaded}
\begin{Highlighting}[]
\DecValTok{1} \OperatorTok{!=}\StringTok{ }\DecValTok{2} \CommentTok{#not equal}
\end{Highlighting}
\end{Shaded}

\begin{verbatim}
## [1] TRUE
\end{verbatim}

\begin{center}\rule{0.5\linewidth}{\linethickness}\end{center}

\hypertarget{logical-operators-1}{%
\subsection{Logical operators}\label{logical-operators-1}}

\begin{Shaded}
\begin{Highlighting}[]
\DecValTok{2} \OperatorTok{>}\StringTok{ }\DecValTok{1} \CommentTok{#greater than}
\end{Highlighting}
\end{Shaded}

\begin{verbatim}
## [1] TRUE
\end{verbatim}

\begin{Shaded}
\begin{Highlighting}[]
\DecValTok{1} \OperatorTok{<}\StringTok{ }\DecValTok{2} \CommentTok{#less than}
\end{Highlighting}
\end{Shaded}

\begin{verbatim}
## [1] TRUE
\end{verbatim}

\begin{Shaded}
\begin{Highlighting}[]
\DecValTok{1} \OperatorTok{==}\StringTok{ }\DecValTok{1} \OperatorTok{|}\StringTok{ }\DecValTok{1} \OperatorTok{==}\StringTok{ }\DecValTok{2} \CommentTok{#or}
\end{Highlighting}
\end{Shaded}

\begin{verbatim}
## [1] TRUE
\end{verbatim}

\begin{Shaded}
\begin{Highlighting}[]
\DecValTok{1} \OperatorTok{==}\StringTok{ }\DecValTok{1} \OperatorTok{&}\StringTok{ }\DecValTok{1} \OperatorTok{==}\StringTok{ }\DecValTok{2} \CommentTok{#and}
\end{Highlighting}
\end{Shaded}

\begin{verbatim}
## [1] FALSE
\end{verbatim}

\begin{center}\rule{0.5\linewidth}{\linethickness}\end{center}

\hypertarget{variable-assignment}{%
\subsection{Variable assignment}\label{variable-assignment}}

\begin{itemize}
\tightlist
\item
  Assigning variables allows you store values for later use
\end{itemize}

--

\begin{Shaded}
\begin{Highlighting}[]
\NormalTok{variable_}\DecValTok{1}\NormalTok{ <-}\StringTok{ }\DecValTok{5}
\NormalTok{variable_}\DecValTok{2}\NormalTok{ <-}\StringTok{ }\DecValTok{10}

\NormalTok{variable_}\DecValTok{3}\NormalTok{ <-}\StringTok{ }\NormalTok{variable_}\DecValTok{2}\OperatorTok{/}\NormalTok{variable_}\DecValTok{1}
\NormalTok{variable_}\DecValTok{3}
\end{Highlighting}
\end{Shaded}

\begin{verbatim}
## [1] 2
\end{verbatim}

--

\begin{itemize}
\tightlist
\item
  For variable assignment, the \texttt{\textless{}-} and \texttt{=}
  operators can be used interchangeably
\item
  But it's better to use \texttt{\textless{}-}
\end{itemize}

--

\begin{itemize}
\tightlist
\item
  Note: the value of a variable is not printed when it is assigned
  (i.e.~when we assign \texttt{5} to \texttt{variable\_1}, we don't get
  any output in the console)
\end{itemize}

\begin{center}\rule{0.5\linewidth}{\linethickness}\end{center}

\hypertarget{variable-assigment}{%
\subsection{Variable assigment}\label{variable-assigment}}

\begin{itemize}
\tightlist
\item
  Variables can also be modified after they've been assigned
\end{itemize}

--

\begin{Shaded}
\begin{Highlighting}[]
\NormalTok{variable_}\DecValTok{1}\NormalTok{ <-}\StringTok{ }\DecValTok{1}
\NormalTok{variable_}\DecValTok{1}
\end{Highlighting}
\end{Shaded}

\begin{verbatim}
## [1] 1
\end{verbatim}

\begin{Shaded}
\begin{Highlighting}[]
\NormalTok{variable_}\DecValTok{1}\NormalTok{ <-}\StringTok{ "hello world"}
\NormalTok{variable_}\DecValTok{1}
\end{Highlighting}
\end{Shaded}

\begin{verbatim}
## [1] "hello world"
\end{verbatim}

\begin{center}\rule{0.5\linewidth}{\linethickness}\end{center}

\hypertarget{data-types}{%
\subsection{Data types}\label{data-types}}

\begin{itemize}
\tightlist
\item
  Data comes in lots of different forms
\end{itemize}

--

\begin{itemize}
\tightlist
\item
  Is ``True'' equal to TRUE?
\end{itemize}

--

\begin{itemize}
\tightlist
\item
  The main data types are:
\end{itemize}

--

\begin{itemize}
\tightlist
\item
  logical; TRUE, FALSE
\end{itemize}

--

\begin{itemize}
\tightlist
\item
  numeric; 12.5, 1, 999
\end{itemize}

--

\begin{itemize}
\tightlist
\item
  integer; 2L, 34L, 1294L
\end{itemize}

--

\begin{itemize}
\tightlist
\item
  character; ``hello world'', ``True''
\end{itemize}

--

\begin{itemize}
\tightlist
\item
  dates; 2019-06-01
\end{itemize}

--

\begin{itemize}
\tightlist
\item
  datetimes; 2019-06-01 12:00:00
\end{itemize}

\begin{center}\rule{0.5\linewidth}{\linethickness}\end{center}

\hypertarget{data-types-1}{%
\subsection{Data types}\label{data-types-1}}

\begin{itemize}
\item ~
  \hypertarget{assigning-the-right-data-type-is-important-as-it-determines-how-the-data-is-stored-and-how-data-can-be-manipulated}{%
  \subsection{Assigning the right data type is important, as it
  determines how the data is stored and how data can be
  manipulated}\label{assigning-the-right-data-type-is-important-as-it-determines-how-the-data-is-stored-and-how-data-can-be-manipulated}}
\end{itemize}

\begin{Shaded}
\begin{Highlighting}[]
\NormalTok{variable_}\DecValTok{1}\NormalTok{ <-}\StringTok{ }\DecValTok{5}
\NormalTok{variable_}\DecValTok{2}\NormalTok{ <-}\StringTok{ "5"}

\NormalTok{variable_}\DecValTok{1} \OperatorTok{+}\StringTok{ }\NormalTok{variable_}\DecValTok{2}
\end{Highlighting}
\end{Shaded}

\begin{verbatim}
## Error in variable_1 + variable_2: non-numeric argument to binary operator
\end{verbatim}

\begin{center}\rule{0.5\linewidth}{\linethickness}\end{center}

\hypertarget{data-types-2}{%
\subsection{Data types}\label{data-types-2}}

\begin{itemize}
\item ~
  \hypertarget{r-will-automatically-decide-the-data-type-when-you-assign-a-variable-but-you-can-force-r-to-store-the-value-as-a-different-data-type-using-the-as.xxxxx-functions}{%
  \subsection{R will automatically decide the data type when you assign
  a variable, but you can force R to store the value as a different data
  type using the as.xxxxx
  functions}\label{r-will-automatically-decide-the-data-type-when-you-assign-a-variable-but-you-can-force-r-to-store-the-value-as-a-different-data-type-using-the-as.xxxxx-functions}}
\end{itemize}

\begin{Shaded}
\begin{Highlighting}[]
\NormalTok{variable_}\DecValTok{1}\NormalTok{ <-}\StringTok{ }\KeywordTok{as.numeric}\NormalTok{(}\StringTok{"5"}\NormalTok{)}
\NormalTok{variable_}\DecValTok{2}\NormalTok{ <-}\StringTok{ }\KeywordTok{as.integer}\NormalTok{(}\StringTok{"5"}\NormalTok{)}

\NormalTok{variable_}\DecValTok{1} \OperatorTok{+}\StringTok{ }\NormalTok{variable_}\DecValTok{2}
\end{Highlighting}
\end{Shaded}

\begin{verbatim}
## [1] 10
\end{verbatim}

--

\begin{itemize}
\tightlist
\item
  But the value has to be coercible in the first place!
\end{itemize}

--

\begin{Shaded}
\begin{Highlighting}[]
\NormalTok{variable_}\DecValTok{1}\NormalTok{ <-}\StringTok{ }\KeywordTok{as.numeric}\NormalTok{(}\StringTok{"hello world"}\NormalTok{)}
\end{Highlighting}
\end{Shaded}

\begin{verbatim}
## Warning: NAs introduced by coercion
\end{verbatim}

\begin{Shaded}
\begin{Highlighting}[]
\NormalTok{variable_}\DecValTok{1}
\end{Highlighting}
\end{Shaded}

\begin{verbatim}
## [1] NA
\end{verbatim}

\begin{center}\rule{0.5\linewidth}{\linethickness}\end{center}

\hypertarget{data-types---factors}{%
\subsection{Data types - Factors}\label{data-types---factors}}

\begin{itemize}
\tightlist
\item
  There is another data type which is fairly common in R called a factor
\end{itemize}

--

\begin{itemize}
\tightlist
\item
  A factor is made up of one or more levels that represent some form of
  grouping
\end{itemize}

--

\begin{itemize}
\tightlist
\item
  For example, if we had a dataframe containing firms from different
  division, we might store ``Lead Division'' as a factor, with the
  levels Banking, Insurance, Investment, Fiduciary
\end{itemize}

--

\begin{itemize}
\tightlist
\item
  Note: when importing data into R, R will automatically try and convert
  columns containing strings into factors (unless you specify otherwise)
\end{itemize}

\begin{center}\rule{0.5\linewidth}{\linethickness}\end{center}

\hypertarget{data-structures}{%
\subsection{Data structures}\label{data-structures}}

\begin{itemize}
\tightlist
\item
  Values in R need to be stored in a specific way
\item
  There are 4 main data structures in R:
\end{itemize}

\begin{center}\rule{0.5\linewidth}{\linethickness}\end{center}

\hypertarget{data-structures---vectors}{%
\subsection{Data structures - Vectors}\label{data-structures---vectors}}

\begin{itemize}
\tightlist
\item
  Vectors
\end{itemize}

--

\begin{itemize}
\tightlist
\item
  Vectors in R are arrays of data in one dimension
\end{itemize}

--

\begin{itemize}
\tightlist
\item
  They are atomic (not recursive)*, but can have named values
\end{itemize}

--

\begin{itemize}
\tightlist
\item
  Even variables with a single value will be stored as a vector with
  length 1
\end{itemize}

--

\begin{Shaded}
\begin{Highlighting}[]
\NormalTok{vector_}\DecValTok{1}\NormalTok{ <-}\StringTok{ }\KeywordTok{c}\NormalTok{(}\DecValTok{1}\NormalTok{,}\DecValTok{2}\NormalTok{,}\DecValTok{3}\NormalTok{,}\DecValTok{4}\NormalTok{,}\DecValTok{5}\NormalTok{,}\DecValTok{6}\NormalTok{)}
\NormalTok{vector_}\DecValTok{1}
\end{Highlighting}
\end{Shaded}

\begin{verbatim}
## [1] 1 2 3 4 5 6
\end{verbatim}

\begin{Shaded}
\begin{Highlighting}[]
\NormalTok{vector_}\DecValTok{2}\NormalTok{ <-}\StringTok{ }\KeywordTok{c}\NormalTok{(}\StringTok{"first_value"}\NormalTok{ =}\StringTok{ }\DecValTok{1}\NormalTok{)}
\NormalTok{vector_}\DecValTok{2}
\end{Highlighting}
\end{Shaded}

\begin{verbatim}
## first_value 
##           1
\end{verbatim}

\begin{itemize}
\tightlist
\item
  *More on the difference between atomic and recursive later ---
\end{itemize}

\hypertarget{data-structures---matrices}{%
\subsection{Data structures -
Matrices}\label{data-structures---matrices}}

\begin{itemize}
\tightlist
\item
  Matrices
\end{itemize}

--

\begin{itemize}
\tightlist
\item
  Matrices are two-dimensional arrays that hold values of the same type
\end{itemize}

--

\begin{itemize}
\tightlist
\item
  Matrices can have named rows or columns, but they cannot be subsetted
  by those names (more later on)
\end{itemize}

--

\begin{Shaded}
\begin{Highlighting}[]
\NormalTok{matrix_}\DecValTok{1}\NormalTok{ <-}\StringTok{ }\KeywordTok{matrix}\NormalTok{(}\KeywordTok{c}\NormalTok{(}\DecValTok{1}\NormalTok{,}\DecValTok{2}\NormalTok{, }\DecValTok{3}\NormalTok{,}\DecValTok{4}\NormalTok{), }\DataTypeTok{nrow =} \DecValTok{2}\NormalTok{, }\DataTypeTok{ncol =} \DecValTok{2}\NormalTok{)}
\KeywordTok{print}\NormalTok{(matrix_}\DecValTok{1}\NormalTok{)}
\end{Highlighting}
\end{Shaded}

\begin{verbatim}
##      [,1] [,2]
## [1,]    1    3
## [2,]    2    4
\end{verbatim}

\begin{center}\rule{0.5\linewidth}{\linethickness}\end{center}

\hypertarget{data-structures---data-frames}{%
\subsection{Data structures - Data
frames}\label{data-structures---data-frames}}

\begin{itemize}
\tightlist
\item
  Data frames

  \begin{itemize}
  \tightlist
  \item
    Most data you'll be working with in R will be held in a data frame
  \end{itemize}
\end{itemize}

--

\begin{itemize}
\tightlist
\item
  Data frames are two dimensional arrays that can hold values of
  different types and have named columns and rows
\end{itemize}

--

\begin{Shaded}
\begin{Highlighting}[]
\NormalTok{dataframe1 <-}\StringTok{ }\KeywordTok{data.frame}\NormalTok{(}\DataTypeTok{numeric_col =} \KeywordTok{c}\NormalTok{(}\DecValTok{1}\NormalTok{,}\DecValTok{2}\NormalTok{),}
                         \DataTypeTok{character_col =} \KeywordTok{c}\NormalTok{(}\StringTok{"hello"}\NormalTok{, }\StringTok{"world"}\NormalTok{),}
                         \DataTypeTok{stringsAsFactors =} \OtherTok{FALSE}\NormalTok{)}
\NormalTok{dataframe1}
\end{Highlighting}
\end{Shaded}

\begin{verbatim}
##   numeric_col character_col
## 1           1         hello
## 2           2         world
\end{verbatim}

\begin{itemize}
\tightlist
\item
  Important notes:
\end{itemize}

--

\begin{itemize}
\tightlist
\item
  Values in the same column must be of the same type
\end{itemize}

--

\begin{itemize}
\tightlist
\item
  Each column must have the same number of rows
\end{itemize}

\begin{center}\rule{0.5\linewidth}{\linethickness}\end{center}

\hypertarget{data-structures---lists}{%
\subsection{Data structures - Lists}\label{data-structures---lists}}

\begin{itemize}
\tightlist
\item
  Lists

  \begin{itemize}
  \tightlist
  \item
    Lists are similar to vectors in that they can contain multiple named
    values
  \end{itemize}
\end{itemize}

--

\begin{itemize}
\tightlist
\item
  However, lists can contain values of any type, including other lists
\end{itemize}

--

\begin{itemize}
\tightlist
\item
  This means you can have a list of data frames, or a list of lists of
  data frames!
\end{itemize}

--

\begin{Shaded}
\begin{Highlighting}[]
\NormalTok{list_}\DecValTok{1}\NormalTok{ <-}\StringTok{ }\KeywordTok{list}\NormalTok{(}\DataTypeTok{numbers =} \KeywordTok{c}\NormalTok{(}\DecValTok{1}\NormalTok{,}\DecValTok{2}\NormalTok{,}\DecValTok{3}\NormalTok{,}\DecValTok{4}\NormalTok{),}
               \DataTypeTok{characters =}\NormalTok{ (}\KeywordTok{c}\NormalTok{(}\StringTok{"hello"}\NormalTok{, }\StringTok{"world"}\NormalTok{))}
\NormalTok{)}
\NormalTok{list_}\DecValTok{1}
\end{Highlighting}
\end{Shaded}

\begin{verbatim}
## $numbers
## [1] 1 2 3 4
## 
## $characters
## [1] "hello" "world"
\end{verbatim}

\begin{center}\rule{0.5\linewidth}{\linethickness}\end{center}

\hypertarget{data-structures---lists-1}{%
\subsection{Data structures - Lists}\label{data-structures---lists-1}}

\begin{itemize}
\tightlist
\item
  Due to this difference between lists and vectors, we refer to a list
  as being \emph{recursive} and vectors as \emph{atomic}
\end{itemize}

--

\begin{itemize}
\tightlist
\item
  Atomic

  \begin{itemize}
  \tightlist
  \item
    ``of or forming a single irreducible unit or component in a larger
    system.''
  \end{itemize}
\end{itemize}

--

\begin{itemize}
\tightlist
\item
  Cannot hold objects of their own type (i.e.~you can't have a vector
  object in a vector)
\end{itemize}

--

\begin{itemize}
\tightlist
\item
  Recursive

  \begin{itemize}
  \tightlist
  \item
    ``characterized by recurrence or repetition.''
  \end{itemize}
\end{itemize}

--

\begin{itemize}
\tightlist
\item
  Can hold objects of their own type (i.e you can have a list, in a
  list, in a list\ldots)
\end{itemize}

\begin{center}\rule{0.5\linewidth}{\linethickness}\end{center}

\hypertarget{subsetting-data-structures}{%
\subsection{Subsetting data
structures}\label{subsetting-data-structures}}

\begin{itemize}
\tightlist
\item
  More often than not, you'll only want to access one or more of the
  values in a data structure
\end{itemize}

--

\begin{itemize}
\tightlist
\item
  For example, you may only want to see the first 10 values of a very
  long vector
\end{itemize}

--

\begin{itemize}
\tightlist
\item
  For vectors, this is done by suffixing the variable with a {[}{]}
  containing the index or indices of values you want
\end{itemize}

\begin{Shaded}
\begin{Highlighting}[]
\NormalTok{values <-}\StringTok{ }\KeywordTok{c}\NormalTok{(}\DecValTok{6}\NormalTok{,}\DecValTok{45}\NormalTok{,}\DecValTok{7}\NormalTok{,}\DecValTok{54}\NormalTok{,}\DecValTok{99}\NormalTok{,}\DecValTok{31}\NormalTok{,}\DecValTok{12}\NormalTok{,}\DecValTok{15}\NormalTok{,}\DecValTok{67}\NormalTok{,}\DecValTok{100}\NormalTok{)}
\NormalTok{values[}\DecValTok{1}\NormalTok{]}
\end{Highlighting}
\end{Shaded}

\begin{verbatim}
## [1] 6
\end{verbatim}

\begin{center}\rule{0.5\linewidth}{\linethickness}\end{center}

\hypertarget{subsetting-data-structures-1}{%
\subsection{Subsetting data
structures}\label{subsetting-data-structures-1}}

\begin{Shaded}
\begin{Highlighting}[]
\NormalTok{values[}\KeywordTok{c}\NormalTok{(}\DecValTok{1}\NormalTok{,}\DecValTok{10}\NormalTok{)]}
\end{Highlighting}
\end{Shaded}

\begin{verbatim}
## [1]   6 100
\end{verbatim}

\begin{Shaded}
\begin{Highlighting}[]
\NormalTok{values[}\DecValTok{1}\OperatorTok{:}\DecValTok{5}\NormalTok{] }\CommentTok{# this is the same as values[c(1,2,3,4,5)]}
\end{Highlighting}
\end{Shaded}

\begin{verbatim}
## [1]  6 45  7 54 99
\end{verbatim}

\begin{center}\rule{0.5\linewidth}{\linethickness}\end{center}

\hypertarget{subsetting-data-structures-2}{%
\subsection{Subsetting data
structures}\label{subsetting-data-structures-2}}

\begin{itemize}
\tightlist
\item
  In R, you can also subset most data structures using {[}{[}{]}{]}
  instead of {[}{]}, but there are two important differences
\end{itemize}

--

\begin{itemize}
\tightlist
\item
  {[}{]} returns the \emph{container} of the value, not the value
\end{itemize}

--

\begin{itemize}
\tightlist
\item
  This will include the name of the value
\end{itemize}

--

\begin{itemize}
\tightlist
\item
  In a list, this will return a named list, rather than a single value
  (an example to follow)
\end{itemize}

--

\begin{itemize}
\tightlist
\item
  {[}{]} can be used with more than one index
  (e.g.~\texttt{{[}c(1,2,3,4,5){]}})
\end{itemize}

--

\begin{itemize}
\tightlist
\item
  {[}{[}{]}{]} returns the \emph{value} at that index, not the container
\end{itemize}

--

\begin{itemize}
\tightlist
\item
  {[}{[}{]}{]} will only accept a single index
\end{itemize}

\begin{center}\rule{0.5\linewidth}{\linethickness}\end{center}

\hypertarget{subsetting-vectors---exercise}{%
\subsection{Subsetting vectors -
exercise}\label{subsetting-vectors---exercise}}

\begin{itemize}
\tightlist
\item
  Create a vector
\end{itemize}

--

\begin{itemize}
\tightlist
\item
  Return the 5th value
\end{itemize}

--

\begin{itemize}
\tightlist
\item
  Return the 1st and 3rd value
\end{itemize}

--

\begin{itemize}
\tightlist
\item
  Return the 1st, 2nd, 3rd, 4th, 5th and 7th value
\end{itemize}

--

\begin{Shaded}
\begin{Highlighting}[]
\NormalTok{vector1 <-}\StringTok{ }\KeywordTok{c}\NormalTok{(}\DecValTok{1}\NormalTok{,}\DecValTok{5}\NormalTok{,}\DecValTok{8}\NormalTok{,}\DecValTok{9}\NormalTok{,}\DecValTok{67}\NormalTok{,}\DecValTok{5}\NormalTok{,}\DecValTok{3}\NormalTok{,}\DecValTok{2}\NormalTok{,}\DecValTok{5}\NormalTok{,}\DecValTok{6}\NormalTok{)}
\NormalTok{vector1[[}\DecValTok{5}\NormalTok{]]}
\end{Highlighting}
\end{Shaded}

\begin{verbatim}
## [1] 67
\end{verbatim}

\begin{Shaded}
\begin{Highlighting}[]
\NormalTok{vector1[}\KeywordTok{c}\NormalTok{(}\DecValTok{1}\NormalTok{,}\DecValTok{3}\NormalTok{)]}
\end{Highlighting}
\end{Shaded}

\begin{verbatim}
## [1] 1 8
\end{verbatim}

\begin{Shaded}
\begin{Highlighting}[]
\NormalTok{vector1[}\KeywordTok{c}\NormalTok{(}\DecValTok{1}\OperatorTok{:}\DecValTok{5}\NormalTok{, }\DecValTok{7}\NormalTok{)]}
\end{Highlighting}
\end{Shaded}

\begin{verbatim}
## [1]  1  5  8  9 67  3
\end{verbatim}

\begin{center}\rule{0.5\linewidth}{\linethickness}\end{center}

\hypertarget{subsetting-matrices}{%
\subsection{Subsetting matrices}\label{subsetting-matrices}}

\begin{itemize}
\tightlist
\item
  For data structures with more that one-dimension, we have to provide
  some more information
\end{itemize}

--

\begin{itemize}
\tightlist
\item
  In these cases, we specify an index for each dimension, starting with
  the row
\end{itemize}

\begin{Shaded}
\begin{Highlighting}[]
\NormalTok{matrix_}\DecValTok{1}\NormalTok{ <-}\StringTok{ }\KeywordTok{matrix}\NormalTok{(}\KeywordTok{c}\NormalTok{(}\DecValTok{1}\NormalTok{,}\DecValTok{2}\NormalTok{,}\DecValTok{3}\NormalTok{,}\DecValTok{4}\NormalTok{), }\DataTypeTok{ncol =} \DecValTok{2}\NormalTok{, }\DataTypeTok{nrow =} \DecValTok{2}\NormalTok{)}
\NormalTok{matrix_}\DecValTok{1}\NormalTok{[}\DecValTok{1}\NormalTok{,}\DecValTok{2}\NormalTok{]}
\end{Highlighting}
\end{Shaded}

\begin{verbatim}
## [1] 3
\end{verbatim}

\begin{Shaded}
\begin{Highlighting}[]
\NormalTok{matrix_}\DecValTok{1}\NormalTok{[,}\DecValTok{2}\NormalTok{]}
\end{Highlighting}
\end{Shaded}

\begin{verbatim}
## [1] 3 4
\end{verbatim}

--

\begin{itemize}
\tightlist
\item
  Matrices can also be subsetted by providing the names of the
  row/column instead of the number, or by using {[}{[}{]}{]}
\end{itemize}

\begin{center}\rule{0.5\linewidth}{\linethickness}\end{center}

\hypertarget{subsetting-data-frames}{%
\subsection{Subsetting data frames}\label{subsetting-data-frames}}

\begin{itemize}
\tightlist
\item
  Unlike matrices, data frame columns can be referred to be name using
  the \$ operator
\end{itemize}

--

\begin{itemize}
\tightlist
\item
  You can combine the \$ and {[}{]} operators to access specific values
  in a column
\end{itemize}

--

\begin{Shaded}
\begin{Highlighting}[]
\NormalTok{dataframe_}\DecValTok{1}\NormalTok{ <-}\StringTok{ }\KeywordTok{data.frame}\NormalTok{(}\DataTypeTok{numeric_col =} \KeywordTok{c}\NormalTok{(}\DecValTok{1}\NormalTok{,}\DecValTok{2}\NormalTok{), }\DataTypeTok{character_col =} \KeywordTok{c}\NormalTok{(}\StringTok{"hello"}\NormalTok{, }\StringTok{"world"}\NormalTok{), }\DataTypeTok{stringsAsFactors =} \OtherTok{FALSE}\NormalTok{)}
\NormalTok{dataframe_}\DecValTok{1}\OperatorTok{$}\NormalTok{character_col}
\end{Highlighting}
\end{Shaded}

\begin{verbatim}
## [1] "hello" "world"
\end{verbatim}

\begin{Shaded}
\begin{Highlighting}[]
\NormalTok{dataframe_}\DecValTok{1}\OperatorTok{$}\NormalTok{character_col[}\DecValTok{1}\NormalTok{]}
\end{Highlighting}
\end{Shaded}

\begin{verbatim}
## [1] "hello"
\end{verbatim}

--

\begin{itemize}
\tightlist
\item
  Note: data frames can also be subsetted using the {[}x, y{]} format,
  with names or indices
\end{itemize}

\begin{center}\rule{0.5\linewidth}{\linethickness}\end{center}

\hypertarget{subsetting-dataframes---exercise}{%
\subsection{Subsetting dataframes -
exercise}\label{subsetting-dataframes---exercise}}

\begin{itemize}
\tightlist
\item
  Create a data frame with 2 columns
\end{itemize}

--

\begin{itemize}
\tightlist
\item
  Return one column using the \$ operator
\end{itemize}

--

\begin{itemize}
\tightlist
\item
  Return the other column using the {[},{]} approach
\end{itemize}

--

\begin{Shaded}
\begin{Highlighting}[]
\NormalTok{df1 <-}\StringTok{ }\KeywordTok{data.frame}\NormalTok{(}\DataTypeTok{numeric_col =} \KeywordTok{c}\NormalTok{(}\DecValTok{1}\NormalTok{,}\DecValTok{2}\NormalTok{,}\DecValTok{3}\NormalTok{), }\DataTypeTok{logical_col =} \KeywordTok{c}\NormalTok{(}\OtherTok{TRUE}\NormalTok{, }\OtherTok{FALSE}\NormalTok{, }\OtherTok{TRUE}\NormalTok{), }\DataTypeTok{stringsAsFactors =} \OtherTok{FALSE}\NormalTok{)}
\NormalTok{df1}\OperatorTok{$}\NormalTok{numeric_col[}\DecValTok{1}\NormalTok{]}
\end{Highlighting}
\end{Shaded}

\begin{verbatim}
## [1] 1
\end{verbatim}

\begin{Shaded}
\begin{Highlighting}[]
\NormalTok{df1[,}\DecValTok{2}\NormalTok{]}
\end{Highlighting}
\end{Shaded}

\begin{verbatim}
## [1]  TRUE FALSE  TRUE
\end{verbatim}

\begin{center}\rule{0.5\linewidth}{\linethickness}\end{center}

\hypertarget{subsetting-lists}{%
\subsection{Subsetting lists}\label{subsetting-lists}}

\begin{itemize}
\tightlist
\item
  Lists can also be subsetted using the \$ and {[}{]} operators
\end{itemize}

--

\begin{Shaded}
\begin{Highlighting}[]
\NormalTok{list_}\DecValTok{1}\NormalTok{ <-}\StringTok{ }\KeywordTok{list}\NormalTok{(}\DataTypeTok{numbers =} \KeywordTok{c}\NormalTok{(}\DecValTok{1}\NormalTok{,}\DecValTok{2}\NormalTok{,}\DecValTok{3}\NormalTok{,}\DecValTok{4}\NormalTok{), }\DataTypeTok{characters =}\NormalTok{ (}\KeywordTok{c}\NormalTok{(}\StringTok{"hello"}\NormalTok{, }\StringTok{"world"}\NormalTok{)))}

\NormalTok{list_}\DecValTok{1}\OperatorTok{$}\NormalTok{numbers}
\end{Highlighting}
\end{Shaded}

\begin{verbatim}
## [1] 1 2 3 4
\end{verbatim}

\begin{Shaded}
\begin{Highlighting}[]
\NormalTok{list_}\DecValTok{1}\OperatorTok{$}\NormalTok{characters[}\DecValTok{1}\NormalTok{]}
\end{Highlighting}
\end{Shaded}

\begin{verbatim}
## [1] "hello"
\end{verbatim}

--

\begin{itemize}
\tightlist
\item
  NB: the \texttt{\$} operator cannot be used on atomic data structures
  (like vectors or matrices)
\end{itemize}

\begin{center}\rule{0.5\linewidth}{\linethickness}\end{center}

\hypertarget{subsetting-lists-1}{%
\subsection{Subsetting lists}\label{subsetting-lists-1}}

\begin{itemize}
\tightlist
\item
  Although lists are technically two-dimensional, they do not have rows
  and columns and so cannot be subsetted using the {[}x, y{]} format
\end{itemize}

--

\begin{itemize}
\tightlist
\item
  Instead, lists are subsetted using the {[}{]} and {[}{[}{]}{]}
  operators
\end{itemize}

--

\begin{itemize}
\tightlist
\item
  These operators can then be combined with whatever operator is
  appropriate for the type of structure in the list (e.g.~\$ for a data
  frame, or {[}{]} for a vector)
\end{itemize}

--

\begin{Shaded}
\begin{Highlighting}[]
\NormalTok{list_}\DecValTok{1}\NormalTok{ <-}\StringTok{ }\KeywordTok{list}\NormalTok{(}\DataTypeTok{numbers =} \KeywordTok{c}\NormalTok{(}\DecValTok{1}\NormalTok{,}\DecValTok{2}\NormalTok{,}\DecValTok{3}\NormalTok{,}\DecValTok{4}\NormalTok{), }\DataTypeTok{characters =}\NormalTok{ (}\KeywordTok{c}\NormalTok{(}\StringTok{"hello"}\NormalTok{, }\StringTok{"world"}\NormalTok{)))}
\NormalTok{list_}\DecValTok{1}\NormalTok{[[}\DecValTok{1}\NormalTok{]][}\DecValTok{2}\NormalTok{]}
\end{Highlighting}
\end{Shaded}

\begin{verbatim}
## [1] 2
\end{verbatim}

\begin{center}\rule{0.5\linewidth}{\linethickness}\end{center}

\hypertarget{functions}{%
\subsection{Functions}\label{functions}}

\begin{itemize}
\tightlist
\item
  Any operation, such as loading a dataset or finding the mean, are
  defined as functions in R
\end{itemize}

--

\begin{itemize}
\tightlist
\item
  A function is a set of steps that takes an input, does some form of
  computation, and returns an output
\end{itemize}

--

\begin{itemize}
\tightlist
\item
  For example, the mean function takes values and returns the mean of
  those values
\end{itemize}

--

\begin{Shaded}
\begin{Highlighting}[]
\NormalTok{values <-}\StringTok{ }\KeywordTok{c}\NormalTok{(}\DecValTok{10}\NormalTok{,}\DecValTok{20}\NormalTok{,}\DecValTok{30}\NormalTok{)}
\KeywordTok{mean}\NormalTok{(values)}
\end{Highlighting}
\end{Shaded}

\begin{verbatim}
## [1] 20
\end{verbatim}

--

\begin{itemize}
\tightlist
\item
  You may have noticed that \texttt{c()} also acts as a function,
  creating a vector with the provided values
\end{itemize}

\begin{center}\rule{0.5\linewidth}{\linethickness}\end{center}

\hypertarget{functions---input-parametersarguments}{%
\subsection{Functions - input
parameters/arguments}\label{functions---input-parametersarguments}}

\begin{itemize}
\tightlist
\item
  Almost all functions will require some input to produce an output
\end{itemize}

--

\begin{itemize}
\tightlist
\item
  These are the function's input parameters or arguments
\end{itemize}

--

\begin{itemize}
\tightlist
\item
  For example, in the previous slide, the mean() function required a
  vector of values to calculate the mean
\end{itemize}

\begin{center}\rule{0.5\linewidth}{\linethickness}\end{center}

\hypertarget{functions---input-parametersarguments-1}{%
\subsection{Functions - input
parameters/arguments}\label{functions---input-parametersarguments-1}}

\begin{itemize}
\tightlist
\item
  Some functions will have optional input parameters
\end{itemize}

--

\begin{itemize}
\tightlist
\item
  To see what arguments/parameters a function requires, type the
  function name preceded by a ``?'' in the R console
  (e.g.~\texttt{?mean})
\end{itemize}

\begin{center}\rule{0.5\linewidth}{\linethickness}\end{center}

\hypertarget{functions---exercise}{%
\subsection{Functions - exercise}\label{functions---exercise}}

\begin{itemize}
\tightlist
\item
  Find the help page for the \texttt{sample()} function
\end{itemize}

--

\begin{itemize}
\tightlist
\item
  Use the \texttt{sample()} function
\end{itemize}

\begin{center}\rule{0.5\linewidth}{\linethickness}\end{center}

\hypertarget{functions---naming-input-parameters}{%
\subsection{Functions - naming input
parameters}\label{functions---naming-input-parameters}}

\begin{itemize}
\tightlist
\item
  When you provide unnamed values to a function, R will automatically
  assign them to an input argument in the order they appear on the
  function's help page
\end{itemize}

--

\begin{itemize}
\tightlist
\item
  This can be hard to keep track of if you're providing multiple
  arguments
\end{itemize}

--

\begin{itemize}
\tightlist
\item
  Best practice, therefore, is to be explicit when providing your input
  arguments
\end{itemize}

--

\begin{itemize}
\tightlist
\item
  Let's use the \texttt{substr()} function as an example
\end{itemize}

\begin{center}\rule{0.5\linewidth}{\linethickness}\end{center}

\hypertarget{functions---naming-input-parameters-1}{%
\subsection{Functions - naming input
parameters}\label{functions---naming-input-parameters-1}}

\begin{itemize}
\tightlist
\item
  The \texttt{substr()} function returns part of a string from a start
  and stop index
\end{itemize}

--

\begin{Shaded}
\begin{Highlighting}[]
\NormalTok{string1 <-}\StringTok{ "hello world"}
\KeywordTok{substr}\NormalTok{(string1, }\DecValTok{1}\NormalTok{,}\DecValTok{5}\NormalTok{)}
\end{Highlighting}
\end{Shaded}

\begin{verbatim}
## [1] "hello"
\end{verbatim}

\begin{itemize}
\tightlist
\item
  The \texttt{substr()} function expects 3 arguments;
\end{itemize}

--

\begin{itemize}
\tightlist
\item
  The string (x)
\end{itemize}

--

\begin{itemize}
\tightlist
\item
  The start point (start)
\end{itemize}

--

\begin{itemize}
\tightlist
\item
  The end point (stop)
\end{itemize}

--

\begin{itemize}
\tightlist
\item
  As shown above, we can just pass those arguments without naming them,
  in that order
\end{itemize}

\begin{center}\rule{0.5\linewidth}{\linethickness}\end{center}

\hypertarget{functions---naming-input-parameters-2}{%
\subsection{Functions - naming input
parameters}\label{functions---naming-input-parameters-2}}

\begin{itemize}
\tightlist
\item
  It's best practice, however, to name the arguments as you pass
  them\ldots{}
\end{itemize}

--

\begin{Shaded}
\begin{Highlighting}[]
\NormalTok{string1 <-}\StringTok{ "hello world"}
\KeywordTok{substr}\NormalTok{(}\DataTypeTok{x =}\NormalTok{ string1, }\DataTypeTok{start =} \DecValTok{1}\NormalTok{, }\DataTypeTok{stop =} \DecValTok{5}\NormalTok{)}
\end{Highlighting}
\end{Shaded}

\begin{verbatim}
## [1] "hello"
\end{verbatim}

\begin{center}\rule{0.5\linewidth}{\linethickness}\end{center}

\hypertarget{recap}{%
\subsection{Recap}\label{recap}}

\begin{itemize}
\tightlist
\item
  R is a free statistical programming language
\item
  One of the most popular environments for R is RStudio
\item
  Mathematical operators in R are mostly the same as in Excel (e.g.~+,
  -, *, /\ldots)
\item
  Values in R can take different forms, and it's important to get the
  form correct!
\item
  Values can then be stored in lots of different structures
\item
  These structures can be subsetted using different approaches ({[}{]},
  {[}x,y{]}, \$, {[}{[}{]}{]})
\item
  Almost every action in R is performed via a function
\item
  A function takes an input, does some computation and returns an output
\item
  It's always good to be explicit when providing arguments to functions
\end{itemize}


\end{document}
